\fancyhead[R]{Lecture 2}

Recall that our problem is to solve the following equation as an inverse
problem
\begin{equation*}
    \cala\pa{F,\phi}=0 \Rightarrow K_F\phi=r_F
\end{equation*}
where $K_F$ is a linear operator and $r_F$ is a known function. We will solve this equation in a functional space. We will use the following notation
\section{Linear equation in functional space}
We will introduce the following terminology in the context of functional
spaces:
\begin{itemize}
    \item Spaces
    \item Linear operator
    \item Solution of linear equations
\end{itemize}
\subsection{Spaces}
Let's define a space (of functions) $\cale$ on $\R$ as a linear space if it
satisifies the following properties:
\begin{itemize}
    \item $\forall f,g\in\cale$, $f+g\in\cale$
    \item $\forall f\in\cale$, $\forall \alpha\in\R$, $\alpha f\in\cale$
\end{itemize}
Now let's define a norm on $\cale$ as a function $\norm{\cdot}:\cale\rightarrow\R$ such that
\begin{itemize}
    \item $\norm{f}\geq 0$ and $\norm{f}=0$ if and only if $f=0$
    \item $\norm{\alpha f}=\abs{\alpha}\norm{f}$
    \item $\norm{f+g}\leq \norm{f}+\norm{g}$
\end{itemize}
\begin{definition}[complete space]
    A space $\cale$ is called a complete space if every Cauchy sequence in $\cale$ converges to a limit in $\cale$.
\end{definition}
\begin{definition}[Banach space]
    A space $\cale$ is called a Banach space if it is a complete space with respect to the norm $\norm{\cdot}$.
\end{definition}
\begin{definition}[scalar product]
    A scalar product on $\cale$ is a function $\angs{\cdot,\cdot}:\cale\times\cale\rightarrow\R$ such that
    \begin{itemize}
        \item $\angs{f,g}=\angs{g,f}$
        \item $\angs{\alpha f,g}=\alpha\angs{f,g}$
        \item $\angs{f+g,h}=\angs{f,h}+\angs{g,h}$
        \item $\angs{f,f}\geq 0$ and $\angs{f,f}=0$ if and only if $f=0$
    \end{itemize}
\end{definition}
If $\cale$ is equipped with a scalar product, then it is a Hilbert space.
\begin{definition}[Hilbert space]
    A space $\cale$ is called a Hilbert space if it is a complete space with respect to the norm $\norm{\cdot}$ induced by the scalar product $\angs{\cdot,\cdot}$.
\end{definition}
The relationship between the norm and the scalar product is given by the following equation:
\begin{equation*}
    \norm{f}=\sqrt{\angs{f,f}}
\end{equation*}
\begin{remark}
    A Banach space B is a complete normed vector space. In terms of generality, it lies somewhere in between a metric space M (that has a metric, but no norm) and a Hilbert space H (that has an inner-product, and hence a norm, that in turn induces a metric). See the summary in Table \ref{tab:linear_space}.
\end{remark}
\begin{table}[ht]
    \centering
    \begin{tabular}{|l|c|c|c|c|}
        \hline
        spaces              & metric     & norm       & inner product & complete   \\
        \hline
        metric space        & \checkmark &            &               &            \\
        \hline
        normed space        & \checkmark & \checkmark &               &            \\
        \hline
        inner product space & \checkmark & \checkmark & \checkmark    &            \\
        \hline
        Banach space        & \checkmark & \checkmark &               & \checkmark \\
        \hline
        Hilbert space       & \checkmark & \checkmark & \checkmark    & \checkmark \\
        \hline
    \end{tabular}
    \caption{Summary of linear spaces/vector spaces}
    \label{tab:linear_space}
\end{table}

\begin{example}
    $L^p\pa{\Omega, \F, \mu}$ is a space of functions such that $\int\abs{f}^p<\infty$. It is a Banach space with the norm $\norm{f}_p=\pa{\int\abs{f}^p}^{1/p}$. Also if $\mu$ is a probability measure, then we have the inclusion $L^p\pa{\Omega, \F, \mu}\subset L^q\pa{\Omega, \F, \mu}$ for $p\geq q$.
\end{example}
\begin{definition}[Sobolev space]
    Let $\Omega\subset\R^d$ be an open set. The Sobolev space $W^{k,p}\pa{\Omega}$ is the space of functions $f:\Omega\rightarrow\R$ such that
    \begin{equation*}
        \norm{f}_{W^{k,p}}=\pa{\sum_{\abs{\alpha}\leq k}\int_{\Omega}\abs{\partial^{\alpha}f}^p}^{1/p}<\infty
    \end{equation*}
    where $\alpha$ is a multi-index and $\partial^{\alpha}f$ is a partial derivative of order $\abs{\alpha}$.

\end{definition}

\subsubsection{Subspaces}
\begin{definition}[subspace]
    Let $\cale$ be a space and $\calh$ be a subspace of $\cale$. Then $\calh$ is a subspace of $\cale$ if it satisfies the following properties:
    \begin{itemize}
        \item $\forall f,g\in\calh$, $f+g\in\calh$
        \item $\forall f\in\calh$, $\forall \alpha\in\R$, $\alpha f\in\calh$
    \end{itemize}
\end{definition}
\begin{proposition}
    $\calh$ is closed if for every sequence $\pa{f_n}_{n\in\N}$ in $\calh$ such that $f_n\rightarrow f$ in $\cale$, we have $f\in\calh$.
\end{proposition}
\begin{remark}
    In a finite dimensional space, every subspace is closed. However, in an infinite dimensional space, a subspace can be closed or not.
\end{remark}
\begin{definition}[Orthogonal subspace]
    Let $\cale$ be a space and $\calh$ be a subspace of $\cale$. Then $\calh^{\perp}$ is the orthogonal subspace of $\calh$ if
    \begin{equation*}
        \calh^{\perp}=\setbra{f\in\cale:\angs{f,g}=0,\forall g\in\calh}
    \end{equation*}
\end{definition}
\begin{remark}
    The orthogonal subspace of a subspace is always closed.
\end{remark}

\begin{definition}[Dual of a space]
    Let $\cale$ be a Banach space. The dual space of $\cale$, denoted by $\cale^*$, is the space of all linear functionals on $\cale$. A linear functional is a linear map from $\cale$ to $\R$.
\end{definition}

\begin{definition}[Riesz representation theorem]
    Let $\cale$ be a Banach space. Then, for every $f(\cdot)\in\cale^*$ ($f$ is linear form by definition), there exists a $\psi \in\cale$ such that \begin{equation*}
        f(\phi)=\angs {\phi,\psi},\quad \forall \phi\in\cale
    \end{equation*}
    Also, $L^p\pa{\Omega, \F, \mu}^*=L^q\pa{\Omega, \F, \mu}$ for $1<p<\infty$ and $\frac{1}{p}+\frac{1}{q}=1$.
\end{definition}
\subsubsection{Basis}
\begin{definition}[Basis of a Hilber space]
    Let $\cale$ be a Hilbert space. A set $\setbra{\phi_j}_{j\in J}$ is a basis of $\cale$ if every $f\in\cale$ can be written as
    \begin{equation*}
        f=\sum_{j\in J}c_j\phi_j
    \end{equation*}
    where $c_j\in\R$ and $\sum_{j\in J}\abs{c_j}^2<\infty$.

\end{definition}
The space is separable if it has a countable basis.
\begin{theorem}
    In a Hilbert separable space, there exists a countable orthonormal basis.
    That is for every $f\in\cale$, we have
    \begin{equation*}
        f=\sum_{j=1}^{\infty}c_j\phi_j
    \end{equation*} which is called the Fourier series decomposition.
\end{theorem}
The norm of a function in a Hilbert separable space can be written as
\begin{equation*}
    \norm{f}^2=\sum_{j=1}^{\infty}\abs{c_j}^2 \quad \text{where} \quad c_j=\angs{f,\phi_j}
\end{equation*}

\subsubsection{Projection}
{\color{blue}Projection in probability space}

